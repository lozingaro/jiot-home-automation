In this paper, we proposed a language-based approach for the integration of
disparate IoT platforms. We built our treatment on the Jolie programming
language. This first result is an initial step towards a more comprehensive
solution for IoT ecosystem integration and management. Concretely, we included
in Jolie the support for two of the most widely used IoT protocols. The
inclusion enables Jolie programmers to interact with the majority of present IoT
devices. Summarizing our results: \emph{i}) we included in Jolie the CoAP
application protocol, also extending the Jolie language to support the UDP
transport protocol, \emph{ii}) we added the support for the MQTT protocol and,
in doing so, \emph{iii}) we tackled the challenging problem of mapping the
renowned pattern of request-responses (typical of HTTP and other widely used
protocols) into the publish/subscribe message pattern of MQTT.\@ The mapping
abstracts from peculiarities of MQTT and is applicable to any publish/subscribe
protocol.

Regarding future work, we are currently investigating the integration in Jolie
of more IoT protocols~\cite{7123563}, in order to extend the usability of the
language in the IoT setting.

It would also be interesting to extend not only Jolie practice as we
have done, but also Jolie formal semantics~\cite{Guidi2006}. To this
end we can take ideas from the formal model of IoT systems presented
in~\cite{LaneseBF13}.

%%%%%%%%%%%%%%%
%Another interesting direction comes from the inclusion of publish/subscribe
%protocols in Jolie. Indeed, the publish/subscribe pattern is renowned for
%enabling high scalability of networks, as well as supporting flexible and highly
%dynamic network topologies~\cite{eugster03}. Our intuition is that, besides
%efficiency and scalability, publish/subscribe architectures can achieve a higher
%degree of reliability if programmed using the Jolie language.
%%%%%%%%%%%%%%%
% Moreover, the
% communication abstractions provided by the language avoid the use of callbacks,
% which are difficult to program, debug, and are one of the main causes of errors
% in concurrent and distributed systems~\cite{fukuda15}.

Another interesting direction for future developments is studying how Jolie
can support the testing of IoT technologies, e.g., to test how
different protocol stacks perform over a given IoT topology. Thanks to the
simplicity of changing the combination of the used protocols (application and
transport), experimenters can quickly test many configurations, also enjoying
a more reliable platform to compare them. Indeed, usually even changing one
of the protocols in the configured stack would require an almost complete
rewrite of the logic of network components. Contrarily, in Jolie, this change
just requires an update of the deployment part of programs, leaving the logic
unaffected. Moreover, such an update could even be done programmatically,
making the practice of repeated experimenting on IoT networks easier and more
standardized.

Finally, as future work, we also consider the possibility of developing a
light-weight version of the language, to be used on low-power IoT devices.
Indeed, in this paper, we assumed that these devices are programmed with
low-level languages, since they can support only a very constrained execution
environment. Clearly, letting programmers develop all the components of an IoT
network in the same language would not only ease its implementation but also
testability, deployment, and maintenance. However, achieving such a result
would require a very challenging engineering endeavor.
