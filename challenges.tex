Jolie currently supports some of the main technologies used in SOAs (e.g.,
HTTP, SOAP). However, only a limited amount of IoT devices uses the media and
protocols already supported by Jolie. Indeed, protocols such as
CoAP~\cite{doi:10.17487/RFC7252,coap} and MQTT~\cite{mqtt-v3.1.1,mqtt}, which
are widely used in IoT scenarios, are not implemented in Jolie. Integrating
these protocols, as we have done, is essential to allow Jolie programs to
directly interact with the majority of IoT devices. We note that emerging
frameworks for interoperability, such as the Web of Things~\cite{w3c17}, rely
on the same protocols we mentioned for IoT, thus JIoT is also compliant with
them.
%
However there are some
challenges linked to the integration of these technologies within Jolie:

\begin{itemize}
  \item \textit{lossless vs.\@ lossy protocols} --- In SOAs,
  machine-to-machine communication relies on lossless protocols: there are no
  strict constraints on energy consumption or bandwidth and it is not critical
  how many transport-layer messages are needed to ensure reliable delivery.
  That is not true in IoT networks, where communication is constrained by
  energy consumption, which dictates what technology stack can be used. Indeed,
  many IoT communication technologies, among which the mostly renowned CoAP
  application protocol, rely on the UDP transport protocol --- a connectionless
  protocol that gives no guarantee on the delivery of messages, but allows one
  to limit message exchanges and, by extension, energy and bandwidth
  consumption. Since Jolie assumes lossless communications, the inclusion of
  connectionless protocols in the language requires careful handling to prevent
  misbehaviors;

  \item \textit{point-to-point vs.\@ publish-subscribe} --- The premise of the
  Jolie language is to provide communication constructs that do not depend on a
  specific technology. To do so, the language assumes a point-to-point
  communication abstraction, which is common to many protocols like HTTP and
  CoAP. However, to integrate the MQTT protocol in Jolie, we need to model its
  point-to-point semantics as MQTT publish-subscribe operations. Indeed, Jolie
  already provides language constructs usable with many communication
  protocols, hence the less disruptive approach is to use the same constructs,
  which are designed for a point-to-point setting, also for MQTT.\@ This
  requires to find for each point-to-point construct a corresponding effect in
  the publish-subscribe paradigm. The final result is that the execution of a given Jolie behavior is similar under both point-to-point and publish-subscribe technologies.
  
\end{itemize} 