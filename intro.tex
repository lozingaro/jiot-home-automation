The Internet of Things (IoT) advocates for multi-layered software platforms,
each adopting its own media protocols and data
formats~\cite{GubbiBMP13,Atzori20102787,7123563}.
%
The problem of integrating layers of the same IoT platform, as well as different
IoT vertical solutions, involves many levels of the communication stack,
spanning from link-layer communication technologies, such as ZigBee and WiFi, to
application-layer protocols like HTTP, CoAP~\cite{coap,doi:10.17487/RFC7252},
and MQTT~\cite{mqtt,mqtt-v3.1.1}, reaching the top-most layers of data-format
integration~\cite{Milenkovic:2015:CII:2843962.2822643}.

Technology-wise, architects of IoT platforms can choose between two approaches
at odds. The first approach favors optimal in-layer communications, i.e.,
choosing media protocols and data formats best suited for the interactions
happening among homogeneous elements, e.g., edge devices (connectionless
protocols and binary data formats~\cite{7123563}), mid-tier controllers
(gateways and aggregators on the RESTful stack~\cite{richardson2008restful}), or
Cloud nodes (scalable publish-subscribe message queues~\cite{garg2013apache}).
Following this first approach is optimal for in-layer communication. However, at
the cross-layer level, the heterogeneity and possible incompatibility of the
chosen standards make enforcing integrity within the IoT system complex and the
resulting integration fragile. The second architectural approach favors
cross-layer consistency, enforcing a unique communication stack over a single
IoT platform. Here cross-layer integration is simpler thanks to the adoption of
a single medium and data format. However such enforced uniformity is the main
cause of the phenomenon known as ``IoT island''~\cite{Soursos16,Gojmerac16},
where IoT platforms take the shape of vertical solutions that provide little
support for collaboration and integration with each other. How to overcome this
limitation is currently a hot topic, tackled also by ongoing EU projects, e.g.,
symbIoTe~\cite{Gojmerac16} and bIoTope~\cite{biotope}.

In this paper we tackle the problem of IoT integration (both cross-layer and
cross-platform) following a language-based approach focused on integration at
both the transport (TCP or UDP) and application layer. To reach our goal we do
not start from scratch, but we leverage the work done in the area of
Service-Oriented Architectures (SOAs)~\cite{Erl07} and we build on the Jolie
programming language~\cite{MONTESI200719,MGZ07,MontesiGZ14,jolie-lang}. In
particular, we rely on those abstractions provided by Jolie that \emph{i}) let
different communication protocols seamlessly coexist and interoperate within the
same program and \emph{ii}) let programmers dynamically choose which
communication stack should be used for any given communication.
%
Concretely, we fork the Jolie interpreter---written in Java---into a prototype
called JIoT~\cite{jiot}, standing for ``Jolie for IoT''.
%
JIoT supports all the protocols already supported by the Jolie interpreter (TCP
at the transport level, and protocols such as SOAP, RMI and HTTP at the
application level), while adding the application-level protocols for IoT, namely
CoAP (and, as a consequence, UDP at the transport level) and MQTT. Notably, when
the application protocol supports different representation formats (such as
JSON, XML, etc.) of the message payload, as in the case of HTTP and CoAP, JIoT,
like Jolie, can automatically marshal and un-marshal data as required.

We structure our presentation as follows. We overview in \cref{sec:approach} our
approach and summarize our contribution in \cref{sec:contribution}. Then, we
discuss the main challenges we faced in our development in
\cref{sec:challenges}, we present how a programmer can use CoAP/UDP
(\cref{sub:coap}) and MQTT (\cref{sub:mqtt}) in JIoT, and we detail our
implementation in \cref{sub:implementation}. We describe, in \cref{sub:case}, a
scenario on Cloud-based home automation where a JIoT architecture coordinates
heterogeneous edge devices. Finally, we position our contribution with respect
to related work in \cref{sec:related} and we draw final remarks in
\cref{conclusion}.

JIoT is available at~\cite{jiot}, released under the LGPL v2.1 license. The code
snippets reported in this paper are based on version 1.2 of JIoT. The
integration of JIoT into the official code-base of the Jolie language is ongoing
work.

\section{Approach Overview}
\label{sec:approach}

Without proper language abstractions, guaranteeing interoperability among
protocols belonging to different technology stacks is highly complex. The
problem is further exacerbated when one has to modify the technology stack used
for some specific interaction. The replacement may be either static, e.g.,
because of the deployment of new, heterogeneous devices in a pre-existing
system, or dynamic, e.g., to support a changing topology of disparate mobile
devices. Contrarily, with JIoT most of the complexity of guaranteeing
interoperability is managed by the language interpreter and hidden from the
programmer.

As an illustrative example of the proposed approach, let us consider a scenario
where we want to integrate two islands of IoT devices, both collecting
temperature data, but relying on different communication stacks, namely HTTP
over TCP and CoAP over UDP.
%
The end goal is to program a collector which receives and aggregates temperature
measurements from both islands.

Following the structure of Jolie programs, the collector programmed in JIoT is
composed of two parts: a \emph{behavior}, specifying the logic of the
elaboration, and a \emph{deployment}, describing in a declarative way how
communication is performed. This separation of concerns is fundamental to let
programmers easily change which communication stack to use, preserving the same
logic for the elaboration.

As an example of program behavior, let us consider the code below, where
\lstinline{main} is the entry point of execution of Jolie programs.
%
\begin{lstlisting}[numbers=left,basicstyle=\footnotesize\ttfamily]
main {
  ...
  receiveTemperature( data );
  ...
}
\end{lstlisting}
%
Above, line 3 contains a reception statement. Receptions in Jolie indicate a
point where the program waits to receive a message. In this case, the collector
waits to receive a temperature measurement on \emph{operation}
\code{receiveTemperature} (an operation in Jolie is an abstraction for
technology-specific concepts such as channels, resources, URLs, \dots). Upon
reception, it stores the retrieved value in variable \code{data}.
%
Besides the logic of computation of the collector, we also need to specify the
deployment, i.e., on which technologies the communication happens; in the
example above, how the collector receives messages from other devices. In Jolie
this information is defined within \emph{ports}. For example, the port to
receive (denoted with keyword \code{inputPort}) HTTP measurements can be defined
as in \cref{lst:ex_port}.
%
\begin{figure}
 \begin{lstlisting}[numbers=left,basicstyle=\ttfamily\footnotesize,caption=Example of interface
and input port in Jolie.,label=lst:ex_port]
interface TemperatureInterface {
 OneWay: receiveTemperature( string )
}

inputPort CollectorPort1 {
 Location: "socket://localhost:8000"
 Protocol: http
 Interfaces: TemperatureInterface
}
\end{lstlisting}
\end{figure}
%
Port \code{CollectorPort1} specifies that the collector expects inbound
communications via \code{Protocol} \code{http} using a TCP/IP socket receiving
at URL \code{"localhost"} on TCP port \code{8000}. A port exposes a set of
operations, collected within a set of \code{Interfaces}. In the example, the
input port \code{CollectorPort1} declares to expose interface
\code{TemperatureInterface}, which is defined at lines 1--3 of Listing of
\code{interface} is declared at lines 1--3 of \cref{lst:ex_port}. The interface
declares the operation \code{receiveTemperature}, including the type of expected
data (\code{string}), as a \code{OneWay} operation, namely an asynchronous
communication that does not require any reply from the collector (except the
acknowledgment automatically provided by the TCP implementation).

Thanks to port \code{CollectorPort1}, the collector can receive data from the
HTTP island. To integrate the second island, we just need to define an
additional port, similar to \code{CollectorPort1}, except for using UDP/IP
datagrams at the transport layer and CoAP~\cite{doi:10.17487/RFC7252,coap} at
the application layer. Hence, the whole code of the collector becomes:
%
\begin{lstlisting}[%
  basicstyle=\footnotesize\ttfamily,
  label=temperature_interface,
  caption=Code of the Collector Example.]
interface TemperatureInterface {
 OneWay: receiveTemperature( string )
}

inputPort CollectorPort1 {
 Location: "socket://localhost:8000"
 Protocol: http
 Interfaces: TemperatureInterface
}

inputPort CollectorPort2 {
 Location: "datagram://localhost:5683"
 Protocol: coap
 Interfaces: TemperatureInterface
}

main {
 ...
 receiveTemperature( data );
 ...
}
\end{lstlisting}
%
The example above highlights how, using the proposed language abstractions, the
programmer can write a unique behavior and exploit it to receive data sent over
heterogeneous technology stacks. Indeed, the \code{receiveTemperature} operation
takes measurements from both the \code{inputPort}s.
%
For instance, if communication over \code{CollectorPort2} fails, port
\code{CollectorPort1} can still receive data.
%
Programmers can also specify elaborations that depend on the used technologies,
by using different operations in different ports. Jolie supports both inbound
and outbound communications, the latter declared with \code{outputPort}s, whose
structure follows that of \code{inputPort}s. Furthermore, the \code{Location}
and \code{Protocol} of \code{outputPort}s can be changed at runtime, enabling
the dynamic selection of the appropriate technologies for each context.

As mentioned, Jolie enforces a strict separation of concerns between behavior,
describing the logic of the application, and deployment, describing the
communication capabilities. The behavior is defined using the typical constructs
of structured sequential programming, communication primitives, and operators to
deal with concurrency (parallel composition and input
choices~\cite{MontesiGZ14}).

Jolie communication primitives comprise two modalities of interaction.

Outbound~\code{OneWay} communications send a message asynchronously, while
\code{RequestResponse} communications send a message and wait for a reply (they
capture the well-known pattern of request-response interactions~\cite{req-rep}).
Dually, inbound \code{OneWay} communications wait to receive a message, without
sending a reply, while inbound \code{RequestResponse}s wait for a message and
send back a reply.

Jolie supports many communication media (via keyword \code{Location}) and data
protocols (via keyword \code{Protocol}) in a simple, uniform way. This is one of
the main features of the Jolie language, and the reason why we base our approach
on it.
%
Each communication port declares the medium and data protocol used to
communicate, hence, to switch to a different technology stack, one just needs to
change the declaration of \code{Location} and \code{Protocol} of a given port.
As expected, the behavior (i.e., the actual logic of computation) of any Jolie
program is unaffected by any change to its ports. Hence, a Jolie program can
provide the same service (i.e., the same behavior) through different media and
protocols just by specifying different deployments. Being born in the field of
SOAs, Jolie supports the main technologies from that area: e.g., communication
media like TCP/IP sockets, Bluetooth L2CAP, Java RMI, and Unix local sockets;
and data protocols like HTTP, JSON-RPC, XML-RPC, SOAP and their respective SSL
versions.

\section{Contribution}
\label{sec:contribution}

To substantiate the effectiveness of our language-based approach to IoT
integration, we add to Jolie support for the main communication stacks used in
the IoT setting. Concretely, the added contribution of JIoT with respect to
Jolie is the integration of two application protocols relevant in the IoT
scenario, namely CoAP~\cite{doi:10.17487/RFC7252,coap} and
MQTT~\cite{mqtt-v3.1.1,mqtt}. Notably, in JIoT the usage of such protocols is
supported by the same linguistic abstractions that Jolie uses for SOAs protocols
such as HTTP and SOAP.

Even if Jolie provides support for the integration of new protocols, when set in
the context of IoT technology, the task is non trivial. Indeed, all the
protocols previously supported by Jolie exploit the same internal interface,
based on two assumptions: \emph{i}) the usage of underlying technologies that
ensure reliable communications and \emph{ii}) a point-to-point communication
pattern.

However, those assumptions do not hold when considering the two IoT technologies
we integrate:

\begin{itemize}

 \item CoAP communications can be unreliable since they are based on UDP
       connectionless datagrams. CoAP provides options for reliable communications,
       however these are usually disabled in an IoT setting, since it is important to
       preserve battery and bandwidth;

 \item MQTT communications are based on the publish-subscribe paradigm, which
       contrasts with the point-to-point paradigm underlying the Jolie communication
       primitives. Hence, we need to define a mapping to express publish-subscribe
       operations in terms of Jolie communication abstractions. In doing so, we need
       to balance two factors: \emph{i}) preserving the simplicity of use of the
       point-to-point communication style and \emph{ii}) capturing the typical
       publish-subscribe flow of communications. Such a mapping is particularly
       challenging in the case of request-response communications. Remarkably, the
       mapping that we present in this work is general and could be used also in other
       contexts.

\end{itemize}

This paper integrates, revises, and extends material
from~\cite{GabbrielliGLZ18}, where we presented, discussed, and provided basic
technical details on the proposed language-based approach to IoT integration.
Main extensions comprise:

\begin{itemize}

 \item advanced technical details on our implementation
       (Section~\ref{sub:implementation}) including:

       \begin{itemize}

        \item a general account on how media and protocols are separated from
              the Jolie interpreter and how they can be developed as independent
              modules;

        \item extensive details on the implementation of UDP, CoAP, and MQTT
              protocols;

       \end{itemize}

 \item a comprehensive case study on a home automation scenario
       (Section~\ref{sub:case}) where we consider:

       \begin{itemize}

        \item local, cross-layer communication among things and mid-tier
              controllers (edge devices and fog nodes);

        \item remote, cross-layer interactions among Cloud nodes and mid-tier
              controllers.

       \end{itemize}

\end{itemize}
%
We conclude this section briefly discussing the current limitations of JIoT
related to its usage in the programming of low-level edge devices---like
Arduinos and other microcontrollers. JIoT supports dynamic scenarios where the
nodes in the network can switch among many technology stacks according to
internal or environmental conditions, such as available energy or quality of
communication. From preliminary discussions with collaborators and IoT
practitioners, we collected positive opinions on the idea of using JIoT for
programming low-level edge devices. Given these positive remarks, we
investigated the feasibility of running JIoT programs over edge devices,
possibly including additional language abstractions to provide low-level access
to in-board sensors and actuators. However, our survey revealed a market of
devices fragmented over incompatible hardware architectures and characterized by
strong constraints over both computational power and energy consumption.
Considering these limitations, we concluded that supporting the execution of
JIoT-like programs over edge devices would require a strong engineering effort.
While this research direction is promising, we deem it non-urgent, since
currently developers tend to program very simple behaviors for edge
devices~\cite{7123563}, which usually capture some data (e.g., through one of
their sensors) and then send them to mid-to-top-tier devices. The latter usually
process and coordinate the flow of data: they have powerful hardware, they
communicate over reliable channels, and they have fewer (if any) constraints
with respect to battery/energy consumption.

Considered the discussion above, in this work we omit the low-level programming
of edge devices and we focus on mid-to-top-tier ones, which can host the JIoT
runtime and which, given their topological context, directly benefit from the
flexibility of the approach.
