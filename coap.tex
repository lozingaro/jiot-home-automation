The \textit{Constrained Application Protocol}
(CoAP)~\cite{coap,doi:10.17487/RFC7252} is a specialized web transfer protocol
for constrained scenarios where nodes have low power and networks are lossy. The
goal of CoAP is to import the widely adopted model of REST
architectures~\cite{fielding00} into an IoT setting, that is, optimizing it for
machine-to-machine applications. In particular, like HTTP,
CoAP makes use of GET, PUT,
POST, and DELETE methods.
%
Following the RFC~\cite{doi:10.17487/RFC7252}, CoAP is implemented on top of the
UDP transport protocol~\cite{UDP}, with optional reliability. Indeed, CoAP
provides two communication modalities: a reliable one, obtained by marking the
message type as confirmable (CON), and an unreliable one, obtained by marking the
message type as non confirmable (NON).

As an example, we consider a scenario with a controller, programmed in
JIoT, that communicates with one of many thermostats
in a home automation scenario.
Thermostats are accessible at the generic address~\code{"coap://localhost/##"}
where~\code{"##"}~is a two-digit number representing the identifier
of a specific device.
Each thermostat accepts two kinds of interactions: a GET request on
URI~\code{"coap://localhost/##/getTemperature"},
that returns the current temperature, and a POST
request on URI \code{"coap://localhost/##/setTemperature"},
that sets the temperature of the HVAC (heating, ventilation, and air
conditioning) system.

We comment below \Cref{lst:coap_example}, where we report the code of a
possible JIoT controller that interacts with a specific thermostat.

\begin{figure}[t]
\begin{lstlisting}[
basicstyle=\footnotesize\ttfamily,
label=lst:coap_example,
caption=JIoT controller communicating over CoAP/UDP.]
type getTmpType: void { .id: string }
type setTmpType: int { .id: string }

interface ThermostatInterface {
 RequestResponse: getTmp( getTmpType )( int )
 OneWay: setTmp( setTmpType )
}

outputPort Thermostat {
 Location: "datagram://localhost:5683"
 Protocol: coap {
  .osc.getTmp << {
   .messageCode = "GET",
   .contentFormat = "text/plain",
   .messageType = "CON",
   .alias = "/%!{id}/getTemperature"
  };
  .osc.setTmp << {
   .messageCode = "POST",
   .messageType = "CON",
   .alias = "/%!{id}/setTemperature"
  }
 }
 Interfaces: ThermostatInterface
}

main {
 getTmp@Thermostat( { .id = "42" } )( temp );
 if ( temp > 27 ) {
  setTmp@Thermostat( 24 { .id = "42" } )
 } else if ( temp < 15 ) {
  setTmp@Thermostat( 22 { .id = "42" } )
 }
}
\end{lstlisting}
\end{figure}

Our scenario includes two CoAP resources, referred to as
~\code{"/getTemperature"} and \code{"/setTemperature"}.
We model them in JIoT at lines 4--7
of \cref{lst:coap_example}, by defining the \code{interface}
\code{ThermostatInterface}, which includes a
\code{RequestResponse} operation \code{getTmp}, representing resource
\code{"/getTemperature"}, and a \code{OneWay} operation
\code{setTmp}, representing resource \code{"/setTemperature"}.
%
By default, we map operation names to resource names, hence in our example we
would need resources named \code{"/getTmp"} and
\code{"/setTmp"}, respectively. However one
can override
this default by defining the coupling of resource names and operations
as desired. This allows programmers to use interfaces as high level
abstractions for interactions,
%the default mapping, defining the coupling of protocol-specific
%concepts (here CoAP resources) and operations inside ports. In this way,
%programmers can define interactions at the high level of abstraction using interfaces,
while the grounding to the specific case is done in the deployment.
%
Here we purposefully choose to use operation names that differ from resource
names to underline that the two concepts are related but loosely coupled.
%
On the one hand the coupling between the name of the resource and the operation can be seen
as a way of quickly binding actions exposed by the CoAP server with operations.
On the other hand decoupling resource names and operations permits to handle
more complex deployments where, for instance, a single operation responds for
different resources.
%
At lines 9--25 we define an \code{outputPort} to interact with the
\code{Thermostat}.
%
At line 10 we specify the \code{Location} of the thermostat. Recalling that
the scheme of the resources of the thermostats is
\code{"coap://localhost/##/..."}, we define the \code{Location} of
the port using the UDP \code{"datagram://"} protocol, followed by the
first part of the resource schema \code{"localhost"} and the UDP port on
which it accepts requests. Here we assume thermostats to use CoAP standard UDP
port, which is \code{"5683"}. Note that, in the \code{Location}, we do
not define the address of a specific thermostat, e.g.,
\code{"datagram://localhost:5683/42"}. On the contrary, we just specify
the generic address to access thermostats in the system, while the specific
binding will be done at runtime, thanks to the \code{.alias} parameter of
the \code{coap} protocol, described later on.

At line 11 we define \code{coap} to be the protocol used by the
\code{outputPort}. At lines 12--22 we specify some parameters of the
\code{coap} protocol --- this matches the standard way in which Jolie defines
parameters for \code{Protocol}s in ports.
Here, we follow the methodology presented in~\cite{montesi16} for the
implementation of the HTTP protocol in Jolie --- indeed CoAP adopts HTTP naming
schema and resource interaction methods. In particular, we draw
from~\cite{montesi16} the parameter prefix \code{.osc}, whose name is the
acronym of ``operation-specific configuration'' and which is used for
configuration parameters related to a specific operation.

In the example, we define \code{.osc} parameters for both operations \code
{getTmp} and \code{setTmp}.\@ At line 13 we specify that the CoAP verb used for
operation \code{getTmp} is \code{"GET"}. At line 14 we define, using the
\code{.contentFormat} parameter, that the encoding of the payload of the message
is in text format. Other accepted values for the \code{.contentFormat} parameter
are \code{"json"} and \code{"xml"}. Marshalling and un-marshalling is automatic
and transparent to the programmer. This feature is enabled by the structure of
Jolie variables, which are always tree-shaped, hence they can be easily
translated into representations based on that shape. At line 15 we set the
\code{.messageType} parameter to \code{"CON"}, that stands for confirmable.
Accepted values for the \code{.messageType} parameter are confirmable and not
confirmable (\code{"NON"}), the latter being the deafault value. In the first
case the sender will receive an acknowledgment message from the receiver, in the
second case it will not. At line 16, following the practice introduced
in~\cite{montesi16}, we specify that \code{getTmp} \code{alias}es a resource
whose path concatenates a static part, given by the \code{Location}, and the
instantiation of the template \lstinline|"/%!{id}/getTemperature"| provided by
protocol parameter \code{.alias}. The template is instantiated using values from
the parameter of the operation invocation in the behavior, e.g., value \code{42}
at line 28\footnote{In Jolie the dot \code{.} defines path traversals inside
trees. Hence, the notation \code{\{.id = 42\}} indicates a tree with an empty
root and a subnode called \code{id}, whose value is \code{42}.}. Hence, the
interpretation of the declaration at line 16 is that, when invoking operation
\code{getTmp} at runtime, the element \code{id} of the invocation will be
removed from the payload and used to form the address of the requested resource.
The aliasing for operation \code{setTmp} (line 21) is similar to that of
\code{getTmp}, while the operation uses verb \code{POST}. Since here the
\code{.contentFormat} parameter is omitted, the default \code{"text/plain"} is
used.

To conclude, we briefly comment the runtime execution of the example, described
in the behavior at lines 28--33. At line 28 the controller invokes operation
\code{getTmp}. Being an outgoing \code{RequestResponse}, the
invocation defines on which port to perform the request (\code{Thermostat})
and presents two pairs of round brackets: the first contains the data for the
request, the second points to the variable that will store the received
response. Recalling the aliasing defined at line 16, at line 28 we define the
value of element \code{id = 42}, thus the URI of the resource invoked at
runtime is \code{"coap://localhost/42/getTemperature"}. Notably, in the
example we hard-coded the \code{id} of the device, however in a more realistic
setting the value of \code{id} would be retrieved dynamically, e.g., as an
execution parameter, from a configuration file or from a database. Once
received, the response from thermostat \code{42} is assigned to variable
\code{temp}. The example concludes with a conditional in which, if the
temperature is above \code{27} degrees (line 29), the thermostat is set to lower room
temperature to \code{24} degrees, while, if the temperature lies below \code
{15} degrees, the thermostat is set to raise the temperature to \code{22}
degrees.

Dually to \code{outputPort}s, \code{inputPort}s allow the programmer
to specify inbound communications. The parameters described above are valid also
for \code{inputPort}s, with the only difference that
\code{messageType} works only for \code{RequestResponse}s, and
specifies whether the communication of the reply is reliable or not.
%
Note that, concerning the \code{.alias} parameter, the template is instantiated
using the address of the incoming communication and the values are inserted
among the elements of the payload.
