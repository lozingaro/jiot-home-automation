\usepackage{bold-extra}

\newcommand{\Jolie}{Jolie}
\newcommand{\Definition}{\noindent\textbf{\emph{Definition}}}
\newcommand{\Implementation}{\noindent\textbf{\emph{Implementation}}}
\newcommand{\Section}{\S}

\newcommand{\citeNeed}{{\color{red}[CitNeed]}}

\definecolor{color:keyword}{rgb}{0.53,0.05,0.05}
\definecolor{color:comment}{rgb}{0.25,0.37,0.75}
\definecolor{color:string}{rgb}{0.87,0.0,0.0}

\lstdefinelanguage{Jolie}{
	morekeywords={csets,type,raw,any,undefined,void,default,if,for,while,spawn,foreach,else,define,main,include,constants,inputPort,outputPort,interface,execution,cset,nullProcess,RequestResponse,OneWay,throw,throws,install,scope,embedded,init,synchronized,global,is_defined,is_int,is_bool,is_long,is_string,bool,long,int,string,double,undef,with,Location,Protocol,Interfaces,Aggregates,Redirects,linkIn,linkOut},
	sensitive=true,
	morecomment=[l]{//},
	morecomment=[s]{/*}{*/},
	morestring=[b]",
	otherkeywords={;,|,@}
}

\lstset{
	language=Jolie,
	mathescape=true,
	resetmargins=true,
	numberstyle=\footnotesize,
	numbers=left,
	numbersep=5pt,
	numberblanklines=true,
	basicstyle=\ttfamily\small,
	tabsize=2,
	%frame=lines,
	commentstyle=\ttfamily\color{color:comment},
	stringstyle=\color{color:string},
	captionpos=b,
	keywordstyle=\bfseries\color{color:keyword},
	showstringspaces=false,
	belowcaptionskip=10mm,
	breaklines=false,
	columns=fullflexible,
	linewidth= 0.8\linewidth
}


\colorlet{punct}{red!60!black}
\definecolor{delim}{RGB}{20,105,176}
\definecolor{keyword}{RGB}{48,0,211}
\colorlet{numb}{magenta!60!black}

\newcommand{\kwd}[1]{{\color{keyword}\textbf{#1}}}
\newcommand{\hid}[1]{{\color{gray}#1}}

\lstdefinelanguage{json}{
    basicstyle=\ttfamily\small,
    commentstyle=\color{color:comment}, % style of comment
    stringstyle=\color{color:string}, % style of strings
    numbers=left,
    numberstyle=\scriptsize,
    stepnumber=1,
    numbersep=8pt,
    showstringspaces=false,
    frame=lines,
    string=[s]{"}{"},
    comment=[l]{:\ "},
    morecomment=[l]{:"},
    literate=
        *{0}{{{\color{numb}0}}}{1}
         {1}{{{\color{numb}1}}}{1}
         {2}{{{\color{numb}2}}}{1}
         {3}{{{\color{numb}3}}}{1}
         {4}{{{\color{numb}4}}}{1}
         {5}{{{\color{numb}5}}}{1}
         {6}{{{\color{numb}6}}}{1}
         {7}{{{\color{numb}7}}}{1}
         {8}{{{\color{numb}8}}}{1}
         {9}{{{\color{numb}9}}}{1}
}

\newcommand{\code}[1]{\lstinline{#1}{}}

\newcommand{\setUmlSeqChartStyle}{
	\tikzset{inststyle/.style={
    rectangle, draw, 
    anchor=west, 
    minimum height=0.8cm, 
    minimum width=1.6cm, 
    fill=white
    %drop shadow={opacity=0,fill=black}]
    }
  }
}

\crefname{figure}{Fig.}{Figs.}
\crefname{lstlisting}{Listing}{Listings}
\crefname{section}{Section}{Sections}
\newcommand*\circled[1]{\tikz[baseline=(char.base)]{
            \node[shape=circle,draw,inner sep=1.3pt] (char) {#1};}}
